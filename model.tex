\newpage
\section{Model}

Given a contract with state variables $\svar$,
the control flow graph of a function $f(\farg)$ 
is the tuple \mbox{$\langle G,\alpha,\omega,\epsilon,\rho\rangle$}.
\mbox{$G=(V,E,\lambda,\mu)$} 
is a node- and edge-labeled directed graph,
where $V$ is the set of CFG blocks, $E$ is the set of transitions 
$\langle V,V\rangle$,
\mbox{$\lambda:V\rightarrow\varphi(\lvar,
\lvar')$}, and \mbox{$\mu:E\rightarrow\varphi(\lvar)$},
where $\lvar$ is the set of local 
variables in $f$, and \mbox{$\lvar'=\{x'\mid x\in\lvar\}$}. 
The CFG blocks $\alpha,\omega\in V$ are respectively the entry 
block and exit block. The CFG block $\epsilon\in V$ is the block
representing an assertion error.
Function \mbox{$\rho:\svar\cup\farg\rightarrow\lvar$} is 
an injection mapping every state variable and function 
argument to a distinct local variable. 

During the execution of $f$ only local variables are manipulated. If
the execution concludes without errors, the state variables
are updated with the respective local values
provided by $\rho$.
%
Function $\lambda$ maps each CFG block to 
the SSA formula modelling such block behaviour, setting for each 
$x\in\lvar$, the value $x'$ resulting from the block
execution.
Function $\mu$ maps each transition to the condition 
under which such transition is taken.
For each \mbox{$\mathsf{assert}(\sigma(\lvar))$} in the 
source code of a block $v\in V$, an edge 
\mbox{$e=\langle v,\epsilon\rangle$}, with \mbox{$\mu(e)=\sigma$}
is added to $E$.

\subsection{Solidity function as CHCs}
Given the control flow graph 
\mbox{$\langle G,\alpha,\omega,\epsilon,\mu\rangle$} 
of a function $f(\farg)$, the set of CHC $\Pi_f$ modelling $f$ 
is constructed
as follows. 
%
For each CFG block $v\in V$, $\pred_v(\lvar)$ 
is a predicate symbol representing the block.

Let $\inP_f(\svar,\farg)$ be the predicate symbol
representing a call to $f(\farg)$ given a contract state $\svar$.
The following CHC is the {\em initial rule} of $f$.
\begin{align}
\label{eq:CHC-initial}
\forall\svar,\farg\cdot\inP_f(\svar,\farg)\bigwedge_{x\in\svar\cup\farg}x=\rho(x)\implies\pred_\alpha(\lvar)
\end{align}

Let $\finP_f(\svar)$ be the predicate symbol
representing the end of a call to $f$, and the consequent
update of state variables. The following CHC is the 
{\em final rule} of $f$.
\begin{align}
\label{eq:CHC-final}
\forall\lvar\cdot\pred_\omega(\lvar)\bigwedge_{x\in\svar}x=\rho(x)\implies\finP_f(\svar)
\end{align}

Finally, the set of rules representing the execution of $f$ is defined as follows.
For each transition $e=\langle v,w\rangle\in E$,
\begin{align}
\label{eq:CHC-transition}
\forall\lvar\cdot\pred_v(\lvar)\wedge\lambda(v)(\lvar,\lvar')\wedge\mu(e)(\lvar)\implies\pred_w(\lvar')	
\end{align}

The set of CHC $\Pi_f$ modelling $f$ is defined as the set
consisting of the initial rule from \cref{eq:CHC-initial},
the final rule from \cref{eq:CHC-final}, and all the 
rules in the set from \cref{eq:CHC-transition}.

\subsection{Solidity contract as CHCs}
This section presents how to construct the set of CHC 
modelling any possible behaviour of a contract.
Let $\conP(\svar)$ the predicate representing the contract state.
Initially, the contract state is set accordingly to the contract
constructor. 
The initial call to the constructor, and the consequent 
contract state update are provided by the following rules.
\begin{align}
\label{eq:CHC-constructor-initial}
\forall\svar,\farg\cdot\top&\implies\inP_\mathsf{constructor}(\svar,\farg)\\
\label{eq:CHC-constructor-final}
\forall\svar\cdot\finP_\mathsf{constructor}(\svar)&\implies\conP(\svar)
\end{align}

Every transaction in the blockchain
involving the contract is in fact a call
to a specific function. The order of such calls is
nondeterministic. Therefore, for each function $f$ in the contract, 
the following rule represents a call to $f$.
\begin{align}
\label{eq:CHC-call-initial}
\forall\svar,\farg\cdot\conP(\svar)\implies\inP_f(\svar,\farg)
\end{align}

Every error-free transaction results in the contract state being updated
according to the called function.
For each function $f$ in the contract, 
the following rule represents a state update
according to $f$.
\begin{align}
\label{eq:CHC-call-final}
\forall\svar\cdot\finP_f(\svar)\implies\conP(\svar)
\end{align}

The set of CHC $\Pi_C$ modelling any possible behaviour of
a contract $C$ is defined as the constructor rules from
\cref{eq:CHC-constructor-initial,eq:CHC-constructor-final},
the transaction rules from 
\cref{eq:CHC-call-initial,eq:CHC-call-final}, 
the rules $\Pi_\mathsf{constructor}$, and
the rules $\Pi_f$ for each function $f$ in the contract.

\subsection{Contract invariants}
\begin{definition}[Contract Invariant]
\label{def:contract-invariant}
	Given a contract $C$, a contract invariant is a formula
	$\varphi(\svar)$ such that \mbox{$\Pi_C\wedge\forall\svar\cdot\conP(\svar)\implies\varphi(\svar)$} is satisfiable.
\end{definition}
%
A contract invariant is a formula over state variables
that is true in any possible state the contract can reach
by submitting any possible transaction.

\subsection{Assertion Error Reachability}

\begin{definition}[Safety Query]
	The safety query of a function control flow graph
	is the CHC
	\mbox{$\forall\lvar\cdot\pred_\epsilon\implies\bot$}.
	The safety query of a contract $C$ is the set $\query_C$ of 
	the safety queries of every function of $C$.
\end{definition}



\begin{definition}[Contract Safety]
	A contract $C$ is safe if and only if
	\mbox{$\Pi_C\wedge\query_C$} is satisfiable.
\end{definition}

The execution of IC3 against \mbox{$\Pi_C\wedge\query_C$}
terminates with either a counterexample consisting of a
finite set of transactions leading to an assertion error,
or a safe inductive invariant proving no assertion error
is reachable. Any formula $\varphi(\svar)$ implied by a
safe inductive invariant is by \cref{def:contract-invariant}
a contract invariant.
